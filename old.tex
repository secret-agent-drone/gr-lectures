\documentclass[a4paper]{article} % A4 paper and 11pt font size

\usepackage{braket}
\usepackage{amsmath}
\usepackage{amssymb}
\usepackage{bm}
\usepackage[utf8]{inputenc}
\usepackage{verbatim}
\usepackage{tikz}
\usepackage{tikz-feynman}
\usepackage{pgfplots}
\usepackage{pgffor}
\usepackage[version-1-compatibility]{siunitx}
\usepackage{fancyhdr}

\usepackage{geometry}

 \geometry{
 a4paper,
 total={210mm,297mm},
 left=28mm,
 right=28mm,
 top=30mm,
 bottom=40mm,
 }


\usepackage{framed}

\usepackage{amssymb} %for Lagrangian L, order O
\usepackage{cancel} %for strikethroughs
\usepackage{slashed} %for Feynman slashes

\newcommand{\me}{\mathrm{e}}
\newcommand{\D}{\mathcal{D}}
\newcommand{\pmx}[1]{\begin{pmatrix}#1\end{pmatrix}}
\newcommand{\m}{\mathcal{M}}
\newcommand{\br}{\mathcal{B}}
\DeclareMathOperator{\im}{Im}
\DeclareMathOperator{\re}{Re}

\newcommand{\bpp}{\(B^0 \to \pi^+\pi^-\)}
\newcommand{\bpk}{\(B^0 \to \phi K_S\)}
\newcommand{\tr}[1]{\text{Tr}\left[#1\right]}
\newcommand{\mev}{\text{ MeV}}
\newcommand{\gev}{\text{ GeV}}

\renewcommand{\c}{\mathcal{C}}
\newcommand{\p}{\mathcal{P}}

\usepackage{gensymb}

%%%%%%%%%%%%%%%CATE'S PREAMBLE BIT%%%%%%%%%%%%%%%%

\usepackage{tikz}
\usetikzlibrary{arrows,shapes}
\usetikzlibrary{trees}
\usetikzlibrary{patterns}
\usetikzlibrary{matrix,arrows} 				% For commutative diagram
											% http://www.felixl.de/commu.pdf
\usetikzlibrary{positioning}				% For "above of=" commands
\usetikzlibrary{calc,through}				% For coordinates
\usetikzlibrary{decorations.pathreplacing}  % For curly braces
% http://www.math.ucla.edu/~getreuer/tikz.html
\usepackage{pgffor}							% For repeating patterns

\usetikzlibrary{decorations.pathmorphing}	% For Feynman Diagrams
\usetikzlibrary{decorations.markings}
\tikzset{
	 >=stealth', %%  Uncomment for more conventional arrows
    vector/.style={decorate, decoration={snake}, draw},
	provector/.style={decorate, decoration={snake,amplitude=2.5pt}, draw},
	antivector/.style={decorate, decoration={snake,amplitude=-2.5pt}, draw},
    fermion/.style={draw=black, postaction={decorate},
        decoration={markings,mark=at position .55 with {\arrow[draw=black]{>}}}},
    fermionbar/.style={draw=black, postaction={decorate},
        decoration={markings,mark=at position .55 with {\arrow[draw=black]{<}}}},
    fermionnoarrow/.style={draw=black},
    gluon/.style={decorate, draw=black,
        decoration={coil,amplitude=4pt, segment length=5pt}},
    scalar/.style={dashed,draw=black, postaction={decorate},
        decoration={markings,mark=at position .55 with {\arrow[draw=black]{>}}}},
    scalarbar/.style={dashed,draw=black, postaction={decorate},
        decoration={markings,mark=at position .55 with {\arrow[draw=black]{<}}}},
    scalarnoarrow/.style={dashed,draw=black},
    electron/.style={draw=black, postaction={decorate},
        decoration={markings,mark=at position .55 with {\arrow[draw=black]{>}}}},
    bigvector/.style={decorate, decoration={snake,amplitude=4pt}, draw},
    arrow/.style={draw=black, postaction={decorate},
        decoration={markings,mark=at position 1 with {\arrow[draw=black]{>}}}},
}

% TIKZ - for block diagrams, 
% from http://www.texample.net/tikz/examples/control-system-principles/
% \usetikzlibrary{shapes,arrows}
\tikzstyle{block} = [draw, rectangle, 
    minimum height=3em, minimum width=6earticlem]

%%%%%%%%%%%%%%END CATE'S PREAMBLE BIT%%%%%%%%%%%%%


\usepackage{fancyhdr}
\usepackage{pdflscape}
\usepackage{bm}

%for side-by-side figures
\usepackage{graphicx}
\usepackage{caption}
\usepackage{subcaption}


%Curly epsilons
\let\oldepsilon\epsilon
\let\epsilon\varepsilon


\setlength{\parindent}{2em}
\setlength{\parskip}{1em}
\renewcommand{\baselinestretch}{1.1}

%----------------------------------------------------------------------------------------
%	TITLE SECTION
%----------------------------------------------------------------------------------------
\setlength\parindent{0pt} % Removes all indentation from paragraphs - comment this line for an assignment with lots of text


\pagenumbering{arabic}
\begin{document}
\pagestyle{empty}

\newcommand{\HRule}{\rule{\linewidth}{0.5mm}}

\begin{titlepage}

    \begin{center}
        \textsc{\large SN: 587623}\\[6cm]

        \HRule \\[0.5cm]
		\Huge \textbf{PHYC90009 Physical Cosmology}\\[0.5cm]
        \huge \textbf{Course Summary}\\[0.5cm] 
        \HRule \\[1.5cm]
        \begin{minipage}{0.4\textwidth}
        \begin{center}

        \large By \\[0.75cm]
        \huge Braden \scshape Moore \\[0.5cm]
        \normalsize \normalfont Master of Science \\
        The University of Melbourne \\

        \end{center}
        \end{minipage}

        \vfill

        \large \today
    \end{center}

\newpage
\end{titlepage}
%----------------------------------------------------------------------------------------
\pagestyle{fancy}
\pagenumbering{gobble}
\tableofcontents
\newpage

\pagenumbering{arabic}
\rfoot{\textsc{Braden Moore, 587623}}
\lfoot{\textsc{\today}}
%\rhead{\textsc{Particle Physics, Part 2: Assignment 3}}
\setcounter{page}{1}
\section{The isotropic, homogeneous universe}
Hello.
 
\section{The Robertson-Walker Metric}
Hello.
 
\section{The Friedmann equations for the dynamics of the Universe}
Hello.
 
\section{Critical density and equations of state}
Hello.
 
\section{Cosmological redshift and time-dilation}
Hello.
 
\section{Distances and the age of the Universe}
Hello.
 
\section{Density fluctuations as the origin of galaxies and the need for an Inflation era}
Hello.
 
\section{The origin of Inflation, and the primordial power-spectrum of density fluctuations}
Hello.
 
\section{The transfer function and the present day linear power-spectrum}
Hello.
 
\section{Matter-radiation decoupling and recombination}
Hello.
 
\section{Evolution of matter and radiation temperatures after decoupling}
Hello.
 
\subsection{Freeze-out}
Hello.
 
\subsection{Baryogenesis}
Hello.
 
\subsection{Big Bang Nucleosynthesis}
Hello.
 
\section{Angular fluctuations in the Cosmic Microwave Background}
Hello.
 
\section{Linear gravitational growth of density perturbations}
Hello.
 
\section{Zeldovich approximation and formation of pancakes}
Hello.
 
\section{Non-linear evolution of a Spherical overdense region}
Hello.
 
\section{The extrapolated linear overdensity at virialisation of a spherical density fluctuation}
Hello.
 
\section{The Press-Schechter mass function and the collapsed mass fraction}
Hello.
 
\section{Cosmology with clusters}
Hello.
 
\section{Linear evolution of baryonic perturbations and the Jeans mass}
Hello.
 
\section{Non-linear collapse of baryonic structures and the cooling timescale}
Hello.
 
\section{The baryon modified Press-Schechter mass function}
Hello.
 
\section{Generation of cosmological HII regions}
Hello.
 
\section{Reionization of the IGM}
Hello.
 
\section{The Gunn-Peterson Trough and the optical depth to scattering of CMB photons}
Hello.
 
\section{Angular momentum in galaxies and the size of galactic disks}
Hello.
 
\section{Clustering of Galaxies}
Hello.
 
\section{Galaxy Bias}
Hello.
 
\subsection{Velocity fields of galaxies}
Hello.
 
\subsection{Halo Occupation}
Hello.
 
\section{Gravitational lensing}
Hello.
 
\section{The gravitational lens cross-section}
Hello.
 
\section{The probability of gravitational lensing in the Universe and magnification bias}
Hello.
 
\end{document}







